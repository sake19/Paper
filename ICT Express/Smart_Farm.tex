\documentclass[11pt]{article}
\usepackage{amsmath}
% use UTF8 encoding
\usepackage[utf8]{inputenc}
% use KoTeX package for Korean
\usepackage{kotex}

\usepackage{hyperref}

\usepackage{graphicx}

\hypersetup{
    colorlinks=true,   
    urlcolor=red,
}

\title{디지털 트윈을 위한 스마트 의사결정 시스템}

\author{Minwoo Jung}

\begin{document}

\maketitle

\section{Introduction}
\indent \\Digital agriculture is the use of technology to improve the convenience and productivity of agricultural activities by digitizing the factors and processes that occur on the farm and in production, distribution, and consumption. Agtech, which includes agricultural biotechnology, precision agriculture, alternative foods, and e-commerce, has led to the formation of new paradigms in the agricultural field.

As information and communication technology is introduced in the field of agriculture, interest in smart farms, which can overcome limitations of space, has increased. Recently, with the introduction of artificial intelligence technology in the field of smart farms, innovative smart farm solutions have been appearing, bringing cost savings and productivity improvements. Figure 1 shows an IoT-based smart farm platform. The smart farm platform uses various sensors installed in agricultural facilities in remote locations to monitor the agricultural environment in real time, and through data analysis, controls the infrastructure through remote controllers. Most smart farm solutions currently being developed are based on greenhouse environments. However, since most farming currently takes place in open fields, it is difficult for users to control the farming environment. Additionally, it is very important to analyze the optimal conditions for smart farms in open field environments in order to construct smart farms. The goal of smart farms is to achieve cost savings and productivity improvements. The most important agricultural activities to consider for smart farms are soil selection and irrigation planning.

Traditional irrigation systems rely on manual intervention by administrators to analyze environmental data and develop irrigation plans, and then execute those plans in a uniform manner. Autonomous irrigation systems can perform irrigation tasks based on pre-set plans without human intervention, but regular monitoring and input by administrators may be necessary to prevent inefficient crop management.

Smart irrigation systems use machine learning to adapt to environmental changes and improve the efficiency and reliability of the system. The key is to build an active irrigation system that can precisely control the water and nutrients needed for crop growth. Soil texture, which is analyzed during the process of analyzing soil conditions, is an important factor in determining the suitability and irrigation plan for crop cultivation in the planning stage of agriculture. Analyzing soil conditions requires specialized knowledge or long-term experiments, but soil moisture sensors can be used to classify soil texture based on changes in soil moisture content.

The paper proposes a smart irrigation system that uses machine learning to determine accurate irrigation schedules and to classify soil texture by analyzing soil moisture changes. The system aims to improve the efficiency and reliability of the irrigation process by actively controlling the system and providing precise control of water and nutrients for crop growth.

\section{Related work}
\indent \\In the past, irrigation systems were not tailored to specific soil conditions, leading to inefficiencies in crop management \cite{smith2008history}. But recent research has aimed to address this by considering the soil moisture levels within a farm and adjusting irrigation schedules and periods based on the characteristics of each management area, in order to improve irrigation efficiency \cite{patel2014improving}. Despite these efforts, there are still limitations in terms of cost and maintenance for implementing these systems on a large scale. Remote sensing technology, such as using satellites, can be costly to operate and maintain \cite{khan2016cost}. Furthermore, the complexity of the system can make it difficult for farmers to operate and manage, which can lead to a lack of adoption \cite{johnson2016challenges}.

In recent years, the Internet of Things (IoT) has provided new opportunities for precision agriculture. Wireless sensor networks have been used to collect soil sensor information, establish and update irrigation plans using embedded systems \cite{Patel2017}, monitor crop growth environments using smartphones \cite{Singh2017}, and control water pumps based on WiFi \cite{Ahmed2019}.

As interest in machine learning increases, various solutions utilizing machine learning technology are emerging in agriculture \cite{Almeida2018}. Researchers are using soil environment data collected by sensors to train machine learning algorithms, such as regression models \cite{Almeida2018_A}. They are also exploring ways to use thermal imaging sensors to measure soil temperature and humidity \cite{almeida2018thermal}, and control a smart irrigation system in real-time through cloud-based technology \cite{ahmed2018cloud}. They are also researching algorithms that can predict soil moisture changes in real-time using machine learning \cite{almeida2018real} and developing algorithms for soil classification using images of soil samples \cite{almeida2018soil}. However, these research efforts are mostly limited to measurement technology and lack control technology, which requires human intervention \cite{almeida2018limitations}. Additionally, using image-based machine learning techniques can be resource-intensive \cite{ahmed2018resource}.

In this paper, we propose a closed-loop control algorithm using machine learning to detect soil conditions and provide efficient irrigation. The proposed algorithm aims to overcome the limitations of existing systems by providing real-time monitoring and control of irrigation schedules based on soil conditions. Furthermore, we propose a soil classification algorithm based on machine learning that can adapt to different soil types and provide appropriate irrigation schedules. They also emphasize the need for further research on management areas and the need for an irrigation system that can adapt to different soil types and provide appropriate irrigation schedules.

\bibliographystyle{IEEEtran}
\bibliography{Smart_Farm_bib}



\end{document}