\documentclass[11pt]{article}
\usepackage{amsmath}
% use UTF8 encoding
\usepackage[utf8]{inputenc}
% use KoTeX package for Korean
\usepackage{kotex}

\usepackage{hyperref}

\usepackage{graphicx}

\hypersetup{
    colorlinks=true,   
    urlcolor=red,
}

\title{디지털 트윈을 위한 스마트 의사결정 시스템}

\author{Minwoo Jung}

\begin{document}

\maketitle

\section{Introduction}

\indent \\농업분야에서 디지털화는 농업 및 식품 생산 시스템에 영향을 미치고, 고급데이터 처리 기술의 적용을 가능하게 한다. 디지털 농업은 농업 생산 시스템 내의 상호관계에 대한 더 깊은 이해와 농업 생산 성과에 대한 결과적인 영향에 대한 더 깊은 이해를 지원하고, 인간의 건강과 웰빙과 같은 사회적 측면과 농업 시스템과 관련된 환경적 측면을 고려한다. 디지털 농업은 농업 환경에서 발생하는 정보를 기반으로 식량 안보, 기후 보호 및 자원 관리와 같은 기존 과제를 해결하는 것을 목표로 한다. 농업분야는 복잡하고 역동적이며 정교한 관리 시스템을 필요로하며, 개인의 경험을 중심으로 한 노하우가 매우 중요하다. 디지털 농업은 기존 아날로그적인 정보를 기반으로 의사결정의 최적화를 제공할 것이다. 혁신적인 정보 통신 기술(ICT), 사물인터넷(IoT), 빅데이터 분석기술, 인공지능(AI)와 같은 기술의 발전이 디지털 농업을 가능하게 하였다. ICT 기술은 농장관리, 생산성, 품질 관리, 식품 공급망 및 의사 결정을 위한 프로세스의 최적화를 지원한다[1-[8,9,10]]. IoT 기술은 실시간 데이터 전송과 모니터링을 가능하게 하여 농장 내 의사 결정 능력을 강화하고 작물 수확량 개선과 손실을 줄여 생산성을 향상 시킬 수 있다. 또한, 클라우드 서버를 구축하여 실시간 데이터 저장하여 빅데이터 분석을 가능하게 하였다. 하지만, 센서 증가와 함께 네트워크 규모가 커지면서 데이터 양이 증가하여 클라우드 서버의 부하가 발생하고 응답 지연시간이 발생한다. 최근, 이러한 네트워크 부하문제를 해결하기 위해서 엣지 컴퓨팅 기술이 등장하였다[1-[13-17]]. 빅데이터 기반 인공지능 기술은 더 정확하고 정확한 농장 모니터링, 데이터 수집 및 분석을 지원하여 센서 및 농장 관리에서 효율성을 향상시켰다. 고급 AI 및 딥러닝 기술은 작물의 건강과 생산성을 모니터링하고 제어할 수 있다[1-[11,12]]. 최근, 앞에서 설명한 다양한 기술들을 기반으로 등장한 디지털 트윈 기술은 가상의 공간을 구축하여 데이터 분석을 통한 시뮬레이션 기반 예측 기술을 통하여 리소스 최적화와 시스템 효율성을 극대화할 수 있다. 디지털 트윈은 우주선 거동을 모니터링하기 위해서 NASA에서 처음 제시하였으며, 물리시스템의 거동을 시뮬레이션하기 위한 물리적 시스템의 가상 또는 디지털 표현으로 정의할 수 있다[1-[21,22]]. 농업분야에서 물리적 시스템은 복잡하고 역동적이며 정교한 관리시스템으로 구성되어 있다. 농업분야의 물리적 구성요소는 동물, 건물, 토양, 기후, 작물 등 다양하게 존재한다. 디지털 트윈은 이러한 물리적 시스템을 기반으로 모델로 구성된 가상환경을 구성한다. 물리세계와 가상세계는 개발된 디지털 트윈의 프로토콜에 따라 무선 및 IoT 기술을 기반으로 데이터를 송수신한다[1-[24,26,27]]. 물리적 시스템에서 수집된 데이터를 분석하여 가상시스템을 업데이트하고, 가상시스템에서 물리시스템으로 피드백한다[1-[25]]. 디지털 트윈의 성능은 데이터의 구조와 크기, 전송속도, 지연시간에 따라 평가된다. 농업분야에서 작물 관리와 생산성은 토양의 품질과 특성에 따라 달라진다. 토양 품질의 모니터링과 관리는 농업분야에서 가장 기본적인 과정이다. 디지털 기술은 농업분야에서 토양을 더 잘 이해할 수 있도록 지원한다. 토양의 수분함량 정보는 관개 효율성을 평가하는데 적절하게 사용할 수 있다. 본 논문에서는 토양의 수분, 온도, 영양분과 관련된 정보를 수집하여 데이터를 분석하고 관개스케줄을 결정할 수 있는 스마트 의사결정 시스템을 제안한다. 스마트 의사결정 시스템은 물 소비 비율을 줄이고, 적절한 영양분을 공급하는 것을 목적으로 한다. 제안하는 시스템의 특징은 토양을 구역화(management zones)하여 단위 구역마다 다른 관개스케줄을 가지는 가변관개시스템을 적용하여 관개 용수 소비를 최적화할 수 있다. 참고문헌[2-[8]]에 따르면, 이러한 가변관개시스템은 관개를 균일하게 적용하는 시스템보다 약 8 \% 이상 물 소비량을 줄일 수 있다고 한다. 또한, 스마트 의사결정 시스템은 사람의 개입없이 토양의 상태에 따라 관개주기(Period)와 관개시간(Duration)을 자율적으로 결정하여 가변할 수 있다. 본 연구는 다음과 같이 구성된다. 2장에서는 연구 배경과 관련연구를 소개한다. 3장에서는 스마트 의사결정 시스템에 대한 아키텍처를 설명하고, 4장에서 실험방법과 결과에 대해서 설명한다. 마지막 5장에서 결론과 향후 연구를 제시한다.

\section{Related work}
\indent \\농업분야에서 농작물의 생장, 수확량, 환경을 정확하게 모니터링하는 기술과 관개, 온습도, 유통을 위한 제어하는 기술은 활발히 연구되고 있다. IoT 기술을 기반으로 디지털 트윈을 구현하기 위해서 분석 및 예측을 위한 시스템 설계 및 알고리즘 구현이 매우 중요하다. 농업 디지털 트윈을 위해서 원격 모니터링 및 제어, 농업 환경 분석 및 예측, 자율 의사결정, 가상화 등 다양한 IT 기술이 필요하다. 농업분야를 위한 원격 모니터링 과 농업환경 분석은 IoT 기술을 활용하여 가장 활발한 연구가 진행되고 있다. [3-[4]]에서 환경 센서를 기반으로 WiFi를 이용하여 원격지에 존재하는 중앙 데이터로 전송하고, 실시간 모니터링 하는 시스템을 구현하였다. 이 논문에서는 토양 수분 센서, 수위센서, 토양 pH센서, 기온센서 및 동작감지센서를 활용하여 데이터를 수집하였으며, 토양 수분 센서와 대기온도 센서를 융합하여 두 가지 알고리즘을 구현하여 보다 효율적인 실시간 모니터링 시스템을 구현하였다. 그들은 [3-[3]]에서 기존 모니터링 시스템에 스마트 대시보드와 스마트폰 제어 시스템을 추가하여 전력 소비량를 모니터링 하였다. [3-[5]]에서 무선 센서 네트워크를 기반으로 데이터를 중앙 서버로 전송하고 정보를 수집하여 실시간 영양상태를 모니터링 하였다. [3-[8]]에서 스마트 농업에서 발생하는 통신, 센서, 하드웨어에서 발생하는 시스템 오류를 해결하였다. 이러한 해결방법을 기반으로 그들은 [3-[10]]에서 SmartFarmNet이라는 스마트 농업 플랫폼을 구현하고, 센서 메타데이터를 검색하여 센서를 발견하는 메커니즘을 제안하였다. [3-[1]]에서 수분 정보를 무선 전송하여 센서의 레벨을 high, middle, low로 구분하여 데이터를 출력하고, 클라이언트에서 제어할 수 있도록 설계하였다. [3-[6]]에서 스마트 농업을 위한 IoT 기반 시맨틱 프레임워크인 Agri-IoT 플랫폼을 제안하였다. 이 플랫폼은 대규모 데이터 분석 및 이벤트 감지 매커니즘을 제공하였다. 이 플랫폼은 네트워크 레이어, 미들 레이어, 애플리케이션 레이어로 계층화하여 시스템을 관리하였다. 앞에서 기술한 연구에서 공통적으로 데이터의 센싱, 수집, 분석과 같은 시스템 아키텍처를 가진다. 하지만, 농업 디지털 트윈을 구현하기 위해서 예측 알고리즘과 자율 의사결정에 대한 연구가 필요하다. 머신러닝 기술에 대한 관심이 급증하면서, 농업분야에서도 머신러닝 기반으로 환경 데이터 분석하여 미래에 발생할 수 있는 이벤트에 대하여 예측할 수 있는 기법에 대한 연구가 진행되고 있다. [4-[8,9]]에서 토양의 온도, 습도와 같은 토양 특성을 예측하거나 식별하기 위한 머신러닝 기술을 적용한 알고리즘을 제안하였으며, [4-[10]]은 pH값과 토양 비옥도 지수 분류와 예측 모델을 제안하였다. [4-[11]]은 토양 비옥도의 중요한 지표인 pH와 Soil Organic matter(SOM)을 예측을 위한 모델을 제안하였다. [4-[12]]에서는 토양의 유기 탄소, 질소, 수분함량을 머신러닝과 선형 다변수 알고리즘으로 분석하여 비교하였다. [4-[13]]은 머신러닝 알고리즘과 자동 회귀 오류 기능을 사용하여 토양 수분함량을 추정하였다. 이러한 연구들은 농업에서 가장 중요한 토양 상태 변화에 대한 예측 알고리즘을 구현하였다. 토양은 모래, 실트, 점토의 함량에 따라 12가지 토성으로 구분한다. 토성에 따라 작물의 생장과 영양 요구도가 달라진다. 토양의 다양한 매개변수에 대한 예측을 통하여 정확한 토성 분류 알고리즘에 대한 연구는 미흡한 실정이다. 농업 디지털 트윈 구현을 위한 자율 의사결정을 위해서 분류된 토성을 기반으로 물과 영양분 공급 주기를 의사결정 하여 제어기에 피드백 주는 closed loop 시스템에 대한 연구가 반드시 필요하다.
IoT 기반 플랫폼은 데이터 분석을 통해 도출된 결과를 기반으로 클라이언트 측면에서 직접 제어해야한다. 하지만, 농업 디지털 트윈을 구현하기 위해서 분석된 결과를 기반으로 자율 의사결정을 통한 제어가 필요하다. 자율 의사결 정을 지원하기 위해서 서버에서 분석된 결과를 원격지에 배치된 센서 모듈로 무선 전송할 수 있는 원격 소프트웨어 업데이트 기술이 필요하다. IoT 플랫폼이 대규모 센서네트워크로 구성되면서, 센서의 소프트웨어 관리에 대한 문제가 야기되면서 원격 소프트웨어 업데이트에 대한 연구들이 지속적으로 진행되었다.
[5-[3]]은 무선센서네트워크에서 소프트웨어 업데이트를 위한 dissemination of new code to the network, traffic reduction, the execution environment of an update in the device, fault detection and recovery와 같은 프레임워크와 프로토콜에 대한 연구를 진행하였다. [5-[2]]에서 네트워크 상에서 센서 노드를 업데이트하는 다양한 문제에 대해서 논의하였다. 소프트웨어 업데이트를 위해서 센서 노드의 메모리 용량, 연산능력, 소비전력, 신뢰성, 네트워크 확장성을 고려해야한다. [5]는 IETF의 RFC9019 표준 아키텍처를 기반으로 OTA 펌웨어 업데이트 알고리즘을 구현하였다. 그들은 LAN과 WAN 환경에서 제안한 솔루션을 검증하였으며, 네트워크 트래픽에 영향을 미치지 않는 것을 증명하였다. 앞에서 설명한 OTA 기술은 센서 노드의 전체 구조가 바뀌었을 때, 일괄적으로 업데이트 하기 위한 아키텍처를 제안하고 있다. 대규모 농경지에서 토양은 구역마다 다른 성질을 가지기 때문에 management zones에 대한 연구를 하고 있다[6]. 이러한 management zones 환경에서 개별 센서 노드마다 업데이트 할 수 있어야 한다. 본 논문에서는 농업 디지털 트윈 플랫폼을 구축하기 위한 핵심 기술인 머신러닝 기반 토성 분류 알고리즘 구현을 통하여 자율 의사결정 시스템을 구축한다. 또한, 원격 소프트웨어 업데이트 알고리즘 구현을 하여 개별 센서 노드를 제어하고, 관개시스템을 구역마다 다르게 제어하는 시스템을 구축한다.

\section{System Architecture}
우리는 머신러닝 기반 의사결정 시스템과 원격 소프트웨어 업데이트 알고리즘으로 핵심 연구 내용을 구분한다. 머신러닝 기반 의사결정시스템은 실시간 센싱 데이터를 수집하여 토성을 분류하고, 토성에 따른 관수 스케줄을 수립한다. 원격 소프트웨어 업데이트 알고리즘은 수정된 관개 스케줄에 따라 관개와 관련된 파라미터를 원격으로 자동 업데이트 할 수 있다. 시스템은 관개제어기, 게이트웨이, 서버로 구성되어 있다. 관개제어기는 토양 센서를 이용한 데이터 수집부, 솔레노이드 밸브를 이용한 관개 제어부, 게이트웨이로 데이터를 전송하기 위한 통신부로 구분된다. 토양 센서는 온습도, NPK(영양분), 전기전도성(EC), pH를 측정하고, 각 센서는 RS-485 MODBUS 프로토콜로 관개제어기와 통신한다. 관개제어기는 토양센서와 솔레노이드 밸브를 통하여 Closed loop을 형성하여 토양의 수분과 영양분에 대한 정보를 피드백하고, 정해진 목표치까지 수분과 영양분을 공급하도록 제어한다. 게이트웨이는 다수의 관개제어기와 BLE 통신을 기반으로 스타 토폴로지로 연결되어 있다. 게이트웨이는 서버로 데이터를 전송하기 위해서 WiFi 모듈이 장착되어 있으며, MQTT 메시징 프로토콜을 이용하여 MQTT Broker와 데이터를 송수신한다.

\subsection{Smart Decision Making}
스마트 의사결정 시스템은 수집된 데이터를 머신러닝을 기반으로 시계열 분석하여 결과를 도출하고, 결과에 따라 관개스케줄을 업데이트한다. 스마트 의사결정을 위한 전제조건은 토성에 따른 토양 수분함량이 다르게 나타나는 현상을 가정한다. 따라서 토양 수분 변화를 모니터링하여 토성을 구분하여 관개주기를 업데이트를 위한 의사결정 기준을 설정할 수 있다. 토성 구분을 위한 수분 변화를 측정하기 위해서 장시간의 수분 변화를 측정해야 하지만 머신러닝 기반 예측 알고리즘을 사용하여 단시간 동안 수집된 데이터를 기반으로 수분 변화를 분석할 수 있다. 

\subsection{Remote Software Update}

\section{Performance Evaluation}

\section{Conclusion}

\end{document}
