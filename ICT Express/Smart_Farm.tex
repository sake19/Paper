\documentclass[11pt]{article}
\usepackage{amsmath}
% use UTF8 encoding
\usepackage[utf8]{inputenc}
% use KoTeX package for Korean
\usepackage{kotex}

\usepackage{hyperref}

\usepackage{graphicx}

\hypersetup{
    colorlinks=true,   
    urlcolor=red,
}

\title{"Smart Irrigation Systems: Utilizing Machine Learning for Real-time Soil Condition Monitoring and Control"}

\author{Minwoo Jung}

\begin{document}

\maketitle

\section{Abstract}
\indent \\The authors highlight the importance of analyzing optimal conditions for smart farms in open field environments and the significance of soil selection and irrigation planning in smart farms. They also mention the limitations of traditional irrigation systems which rely on manual intervention and autonomous irrigation systems which require regular monitoring. The paper proposes a smart irrigation system that uses machine learning to determine accurate irrigation schedules and to classify soil texture by analyzing soil moisture changes, with the goal of improving the efficiency and reliability of the irrigation process. The authors also review related works on precision agriculture and machine learning in agriculture and mention the limitations of existing systems, such as cost and maintenance issues and lack of control technology. They propose a closed-loop control algorithm using machine learning to detect soil conditions and provide efficient irrigation and a soil classification algorithm based on machine learning that can adapt to different soil types.

\section{Introduction}
\indent \\Digital agriculture is the use of technology to improve the convenience and productivity of agricultural activities by digitizing the factors and processes that occur on the farm and in production, distribution, and consumption. Agtech, which includes agricultural biotechnology, precision agriculture, alternative foods, and e-commerce, has led to the formation of new paradigms in the agricultural field.

As information and communication technology is introduced in the field of agriculture, interest in smart farms, which can overcome limitations of space, has increased. Recently, with the introduction of artificial intelligence technology in the field of smart farms, innovative smart farm solutions have been appearing, bringing cost savings and productivity improvements. Figure 1 shows an IoT-based smart farm platform. The smart farm platform uses various sensors installed in agricultural facilities in remote locations to monitor the agricultural environment in real time, and through data analysis, controls the infrastructure through remote controllers. Most smart farm solutions currently being developed are based on greenhouse environments. However, since most farming currently takes place in open fields, it is difficult for users to control the farming environment. Additionally, it is very important to analyze the optimal conditions for smart farms in open field environments in order to construct smart farms. The goal of smart farms is to achieve cost savings and productivity improvements. The most important agricultural activities to consider for smart farms are soil selection and irrigation planning.

Traditional irrigation systems rely on manual intervention by administrators to analyze environmental data and develop irrigation plans, and then execute those plans in a uniform manner. Autonomous irrigation systems can perform irrigation tasks based on pre-set plans without human intervention, but regular monitoring and input by administrators may be necessary to prevent inefficient crop management.

Smart irrigation systems use machine learning to adapt to environmental changes and improve the efficiency and reliability of the system. The key is to build an active irrigation system that can precisely control the water and nutrients needed for crop growth. Soil texture, which is analyzed during the process of analyzing soil conditions, is an important factor in determining the suitability and irrigation plan for crop cultivation in the planning stage of agriculture. Analyzing soil conditions requires specialized knowledge or long-term experiments, but soil moisture sensors can be used to classify soil texture based on changes in soil moisture content.

The paper proposes a smart irrigation system that uses machine learning to determine accurate irrigation schedules and to classify soil texture by analyzing soil moisture changes. The system aims to improve the efficiency and reliability of the irrigation process by actively controlling the system and providing precise control of water and nutrients for crop growth.

\section{Related works}
\indent \\In the past, irrigation systems were not tailored to specific soil conditions, leading to inefficiencies in crop management \cite{smith2008history}. But recent research has aimed to address this by considering the soil moisture levels within a farm and adjusting irrigation schedules and periods based on the characteristics of each management area, in order to improve irrigation efficiency \cite{patel2014improving}. Despite these efforts, there are still limitations in terms of cost and maintenance for implementing these systems on a large scale. Remote sensing technology, such as using satellites, can be costly to operate and maintain \cite{khan2016cost}. Furthermore, the complexity of the system can make it difficult for farmers to operate and manage, which can lead to a lack of adoption \cite{johnson2016challenges}.

In recent years, the Internet of Things (IoT) has provided new opportunities for precision agriculture. Wireless sensor networks have been used to collect soil sensor information, establish and update irrigation plans using embedded systems \cite{Patel2017}, monitor crop growth environments using smartphones \cite{Singh2017}, and control water pumps based on WiFi \cite{Ahmed2019}.

As interest in machine learning increases, various solutions utilizing machine learning technology are emerging in agriculture \cite{Almeida2018}. Researchers are using soil environment data collected by sensors to train machine learning algorithms, such as regression models \cite{Almeida2018_A}. They are also exploring ways to use thermal imaging sensors to measure soil temperature and humidity \cite{almeida2018thermal}, and control a smart irrigation system in real-time through cloud-based technology \cite{ahmed2018cloud}. They are also researching algorithms that can predict soil moisture changes in real-time using machine learning \cite{almeida2018real} and developing algorithms for soil classification using images of soil samples \cite{almeida2018soil}. However, these research efforts are mostly limited to measurement technology and lack control technology, which requires human intervention \cite{almeida2018limitations}. Additionally, using image-based machine learning techniques can be resource-intensive \cite{ahmed2018resource}.

In this paper, we propose a closed-loop control algorithm using machine learning to detect soil conditions and provide efficient irrigation. The proposed algorithm aims to overcome the limitations of existing systems by providing real-time monitoring and control of irrigation schedules based on soil conditions. Furthermore, we propose a soil classification algorithm based on machine learning that can adapt to different soil types and provide appropriate irrigation schedules. They also emphasize the need for further research on management areas and the need for an irrigation system that can adapt to different soil types and provide appropriate irrigation schedules.

\section{System architecture}
\indent \\The proposed smart irrigation system is a closed-loop control system that uses sensors to measure various soil conditions such as temperature, humidity, nutrient content, electrical conductivity, and pH. This data is collected and analyzed by an edge layer for sensor network management and a server layer for data storage and analysis. The system uses this data to activate a solenoid valve to supply moisture and nutrients to the soil when needed, and adjusts the amount based on the soil's classification. It also monitors the soil moisture and nutrient content in real-time using electrical conductivity to ensure optimal irrigation.

\subsection{Sensor layer}
\indent \\The sensor layer of the system is made up of various soil sensors for analyzing the state of the soil and controllers for providing moisture and nutrients. The end devices ensure the reliability of sensor and actuator operation. Each end device can connect to 4 soil sensors and 1 actuator via RS-485 communication. The soil sensors measure temperature, humidity, electrical conductivity, and nutrients. The actuator is equipped with a relay and solenoid valve to control moisture supply. The microcontroller used in the end device is Noridic's nRF52840, which has built-in Bluetooth and a Ubinos operating system for multitasking. The irrigation system uses a venturi to supply moisture and nutrients together. The end device is designed to be watertight when installed on a window and will be optimized through software upgrades and additional sensors to improve the accuracy of soil analysis in the future.

\subsection{Edge layer}
\indent \\The gateway uses BLE and WiFi modules to manage the valve controller and sends collected data to the server. It collects data from the valve controller and converts it to an MQTT message structure before sending it to the server. It operates on a constant 220V power supply and is designed considering power and performance. Additionally, it also has a microphone and infrared sensor for event notifications and theft detection.

\subsection{Server layer}
\indent \\This statement is a concise summary of the implementation of the server layer in a system that uses MQTT to gather data from remote gateways, stores the data in a message queue and a MySQL database, uses machine learning algorithms to classify soil data, and provides a graphical user interface (GUI) for viewing the data without spatial constraints. The GUI is implemented using PHPChart and the deep learning framework Keras on a web server powered by Apache HTTP.

\section{Irrigation control system}
\indent \\The irrigation controller is a system that periodically monitors the moisture level in the soil and delivers water to the plants when the moisture level falls below a predetermined target level. This is accomplished by connecting the irrigation controller and soil sensors through an RS-485 communication network, in a 1:N configuration, where the soil sensors have unique IDs and transmit sensor data to the irrigation controller in a sequential manner at preset intervals.

The irrigation controller is configured as a closed-loop control system that uses sensors and actuators to maintain the target moisture level. A relay is installed between the irrigation controller and the actuator for precise control, and exception handling code is included in the system to detect and analyze any malfunctions of the irrigation controller.

Upon turning on the system, the irrigation controller initializes the input/output interface, sensors, and solenoid valves. Once initialization is complete, the solenoid valves supply moisture for a set period of time. Subsequently, the soil sensors transmit information to the irrigation controller at preset intervals, and the irrigation controller adjusts the solenoid valves to maintain the target moisture level.

The flowchart of the irrigation controller is illustrated in Figure 6.

\section{Soil classification algorithm}
\indent \\The soil's texture can vary greatly and this affects its moisture retention. To address this, we have developed a soil classification algorithm that uses machine learning to analyze long-term moisture data stored on a server. The algorithm employs time series analysis to classify the soil's texture. One of the challenges of determining soil moisture is the need for long-term observation. To overcome this, we have implemented a machine learning algorithm that predicts soil moisture at a specific future point in time, based on the change in soil moisture accumulation over a short period. The soil classification algorithm is then implemented using this moisture prediction as a basis. The experimental groups consisted of clay, sand and loam soil types.

\subsection{Preprocessing}
\indent \\The initial moisture content in soil can vary greatly depending on factors such as sensor location and soil conditions, making it challenging to classify based on absolute values of soil moisture. Additionally, as seen in Figure 7, there is a rapid change in moisture content immediately after watering, requiring a stabilization period. To address these issues, data preprocessing techniques such as removing outliers or normalizing the data are necessary. Data preprocessing is an important step that can significantly impact the performance of a machine learning model.

Normalization is a critical aspect of machine learning, also known as feature scaling or data scaling. Common normalization methods include min-max normalization, Z-score normalization, and standard normalization. Min-max normalization is susceptible to being affected by outliers, so Z-score normalization is often used as an alternative. Z-score normalization is calculated by dividing the difference between a specific value and the mean by the standard deviation. Values that are consistent with the mean are normalized to 0, values less than the mean are negative, and values greater than the mean are positive. The size of the negative and positive values is determined by the standard deviation.

To remove outliers, the change in soil moisture content was first normalized. Normalization allows us to understand the distribution of data based on the mean of the data. The change in soil moisture content was normalized using equation 1, where Xi represents soil moisture, μ represents the average soil moisture, and σ represents the standard deviation. The normalized data was then classified using the Interquartile Range (IQR) method to remove outliers. The IQR method involves dividing the data into sections by sorting the entire data values in order. The entire data is divided into four regions by extracting the Q1, Q3 points and defining the difference between the two points as IQR. The upper and lower boundaries are determined based on Q1, Q3, and data outside the region is removed.

\subsection{Classification algorithm}
\indent \\The preprocessed data was divided into training (70%), validation (20%), and testing (10%) sets for efficient predictions. To maintain the time axis, time series data was used and sequential splits were employed instead of random ones. To prevent overfitting, the validation and testing sets were not utilized during the training process.

Machine learning models for time series data utilize the sliding window technique, which is based on consecutive sample windows to perform predictions. The input window has characteristics such as the width of the input and label windows, and the time offset. This technique is used to compare specific range values of array or list elements.

Time series prediction structures are generally divided into single-step and multi-step models. Single-step models predict a single time step in the future based on current conditions, while multi-step models predict a range of time steps in the future based on current conditions. Common techniques for multi-step predictions include direct multi-step, recursive multi-step, and multiple output. In this research, the recursive multi-step technique was used to predict future changes in water volume based on input data. This technique extends the single time step to a multi-step model by using the prediction from the previous time step as input for predicting the next time step.

For time series prediction, commonly used models are Convolutional Neural Networks (CNNs) and Recurrent Neural Networks (RNNs). While CNNs are typically used for image classification, for time series prediction, 1-dimensional CNNs are utilized. RNNs are often used to learn data that changes over time, such as time series data. However, as the layers between input and output become deeper, the problem of long-term dependency arises, resulting in decreased accuracy. To overcome this problem, the Long Short-Term Memory (LSTM) algorithm was introduced. LSTM uses a memory cell to store and refer to short-term and long-term states in order to solve the problem of long-term dependency, and also allows for faster convergence during learning.

To compare performance, three models were designed, as previously explained. In conclusion, a predictive model was designed to classify asteroids based on their moisture content. The decision boundary, which is the line that separates the different classifications, was determined by analyzing the maximum and minimum values obtained from repeating the training process 100 times. The data for moisture content was found to have a linear separable pattern, and the model was designed to take in various time series data and output predictions that classify the asteroids into 3 different categories.

\section{Experimental result}
\subsection{Irrigation control system}
\indent \\The irrigation controller was installed for both indoor and outdoor use, and the outdoor controller has been functioning properly for an extended period of time. It has been confirmed that the waterproofing is effective during rain. The indoor gateway and the irrigation controller communicate effectively via Bluetooth at an adequate distance. The outdoor controller has been confirmed to collect soil data and ensure system stability. It was observed that the temperature rises and the moisture content decreases linearly in outdoor environments, and in the case of precipitation, the soil moisture rapidly increases and the nutrients are activated, resulting in an increase in data from the NPK sensor. This confirms that sufficient moisture is necessary to sense nutrients.

The irrigation cycle in indoor environments was set to 8 hours, with an irrigation time of 10 seconds. Shortening the irrigation cycle can lead to root rot due to excessive moisture, so a sufficient irrigation cycle is necessary. The system is designed to wait in sleep mode for 8 hours before waking up to compare conditions and operate. The target humidity was set to 50%, and water is supplied during the irrigation time when the humidity is below the target. To allow the sensor to recognize the absorbed water in the soil, a waiting time of 30 seconds was set after supplying water. After the waiting time, the humidity is monitored, and if it is below the target, water is supplied again during the irrigation time. In case of malfunction, the software is designed to reset. The performance of the irrigation system was confirmed by ensuring that the system accurately operates within the set cycle and the target humidity level, and that the changed data is transmitted to the server.

Figure 10 shows data stored on the server through the irrigation system. It was observed that the system regularly supplied moisture when the target moisture level was below, and occasional malfunctions occurred. The malfunctions were identified as being caused by unstable communication conditions. It was also observed that the moisture levels inside and outside were different, and that, as shown in Figure 12, when the temperature rose due to the moisture level being maintained in the plant pot after supplying moisture in the indoor environment, the soil humidity also rose. The data from the NPK sensors was used to check the amount of nutrients, and nutrients were supplied along with moisture during the period when the nutrients decreased. As seen in Figure 11, it was possible to see that the sensor responded after a certain period of time when the nutrients were supplied.

\subsection{Soil classification algorithm}
\indent \\The data, with outliers removed, was applied to the model and validated using CNN, SimpleRNN, and LSTM, which are commonly used in time series analysis. The mean absolute error of the predicted values before and after removing outliers is shown in Figure 12, which demonstrates that the performance of all models has been significantly improved. Preprocessing refers to the processed data, while "Raw" refers to the unprocessed data. The validation data is represented by "V" and the evaluation data is represented by "T".

To classify soil, three types of soil were prepared in containers, and data was extracted under the same conditions by controlling moisture supply. To validate the classification model, evaluation indicators were evaluated based on the confusion matrix for each model. The evaluation indicators of the confusion matrix include accuracy, precision, recall, and F1-Score. These evaluation indicators have both advantages and disadvantages. Accuracy is an indicator that represents the proportion of actual values and predicted values that match in the overall prediction results. Equation 2 is the formula used to calculate accuracy. However, accuracy has the disadvantage of being difficult to properly classify in situations where the probability of negative is too high, which is known as the "accuracy paradox".

Precision, recall, and F1-score are evaluation metrics that are used to evaluate the performance of a classification model. Precision is a measure of how well the model correctly predicts positive cases as positive, while recall is a measure of how well the model detects all the positive cases. F1-score is a combination of precision and recall, and is a good measure of a model's performance when there is a trade-off between precision and recall. In this case, data sets of three types of tones were prepared for classification model validation, and the performance of three models were evaluated using the confusion matrix and the F1-score. The results of the evaluation, as shown in Table 1, indicate that all models have an accuracy of over 90%, and that there is a correlation between recall and precision.

\section{conclusion}
\indent \\In conclusion, the proposed smart irrigation system is a closed-loop control system that utilizes sensors to gather data on various soil conditions, and then delivers moisture and nutrients to the soil as needed. The system is divided into three layers: the sensor layer, the edge layer, and the server layer. The sensor layer is composed of various soil sensors and controllers that provide moisture and nutrients to the soil. The edge layer manages the valve controller and sends collected data to the server. The server layer utilizes MQTT and machine learning algorithms to classify soil data and provides a graphical user interface for viewing the data. Additionally, an irrigation control system and soil classification algorithm have been developed to improve the accuracy of soil analysis and maintain the target moisture level. Furthermore, the system can also be integrated with weather forecasting tools to predict future weather patterns and adjust irrigation accordingly. Overall, this system is designed to optimize irrigation, improve the efficiency of crop growth, and ultimately increase crop yields. This study proposes a feedback-based smart irrigation system that addresses the need for data collection in outdoor fields and uses machine learning to classify soil and analyze suitability for different crops. The closed-loop control environment at the edge level enables accurate water and nutrient supply at the right time. In future research, closed-loop control will be set up to automatically update irrigation schedules and network multiple irrigation controllers to manage large-scale fields. Additionally, the system will be enhanced by incorporating multiple weather forecasting tools to predict future weather patterns and adjust irrigation accordingly. In future work, the system can be enhanced by adding the ability to monitor crop growth and detect any issues early on, allowing farmers to take proactive measures to address them.

\bibliographystyle{IEEEtran}
\bibliography{Smart_Farm_bib}

\end{document}