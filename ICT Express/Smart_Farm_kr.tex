\documentclass[11pt]{article}
\usepackage{amsmath}
% use UTF8 encoding
\usepackage[utf8]{inputenc}
% use KoTeX package for Korean
\usepackage{kotex}

\usepackage{hyperref}

\usepackage{graphicx}

\hypersetup{
    colorlinks=true,   
    urlcolor=red,
}

\title{디지털 트윈을 위한 스마트 의사결정 시스템}

\author{Minwoo Jung}

\begin{document}

\maketitle

\section{Introduction}
\indent \\디지털 농업은 농업 현장에서 발생하는 경험적인 요인과 생산·유통·소비 등에서 발생하는 과정들을 디지털화하여 농업 활동의 편의성과 생산성을 향상시키는 기술이다. Agtech가 등장하면서 농업생명공학기술, 정밀농업, 대체식품, 식품 전자상거래 등 농업 분야에서 새로운 패러다임이 형성되고 있다.

정보통신기술이 농업 분야에 도입되면서, 시공간의 제약을 없앨 수 있는 스마트팜에 대한 관심이 급증하고 있다. 최근, 인공지능 기술이 스마트팜 분야에 도입되면서 혁신적인 스마트팜 솔루션들이 줄줄이 등장하여 비용 절감과 생산성 향상 효과를 가져오고 있다[1]. 그림 1은 IoT 기반 스마트팜 플랫폼을 보여주고 있다. 스마트팜 플랫폼은 원격지에 존재하는 농업시설에 배치된 다양한 센서를 통하여 농업 환경을 실시간 관측하며, 데이터 분석을 통해 원격지의 제어기를 통하여 기반시설을 제어하는 기술이다. 현재, 스마트팜을 위한 대부분 솔루션은 온실 환경을 기반으로 연구되고 있다. 하지만, 현재의 농업은 대부분 노지에서 이루어지기 때문에, 사용자가 농업 환경의 제어를 하기 어렵다는 단점을 가진다. 또한, 온실 환경에서 스마트팜을 구축하기 위해서 노지 환경에서 최적의 조건을 분석하는 과정이 매우 중요하다. 스마트팜은 비용 절감과 생산성 향상이 목적이다. 스마트팜의 목적을 달성하기 위해서 최우선으로 고려해야 하는 농업 활동은 토양선정과 관개 계획 수립이다.

기존 관개 시스템은 환경 데이터를 분석하여 관개 계획을 수립하고, 관리자가 개입하여 획일적인 관개 작업을 수행한다. 자율 관개 시스템은 초기에 설정된 관개 계획에 따라 획일적인 관개를 수행하지만, 비효율적인 작물 관리가 발생할 수 있기 때문에 주기적인 관리자 개입이 필요하다.

스마트 관개 시스템은 환경 변화를 스스로 학습하여 능동적으로 제어하며 시스템의 효율성과 신뢰성을 개선할 수 있다. 핵심은 작물 생육에 필요한 수분과 영양분을 정밀하게 제어할 수 있는 능동형 관개 시스템을 구축하는 것이다. 토양 상태를 분석하는 과정에서 토성(Soil texture)은 농업 계획 수립단계에서 작물 재배의 적합도 및 관개 계획을 판단할 수 있는 중요한 요소이다. 토양 상태를 분석하기 위해서 전문적인 지식이나 장기적인 실험이 필요하지만 수분 센서를 이용하여 토양 수분 함량의 변화를 기반으로 토성을 분류할 수 있다.



\section{Related work}
\indent \\

\end{document}
